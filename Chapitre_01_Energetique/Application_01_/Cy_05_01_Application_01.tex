\documentclass[10pt,fleqn]{article} % Default font size and left-justified equations
\usepackage[%
    pdftitle={Modélisation dynamique},
    pdfauthor={Xavier Pessoles}]{hyperref}

    
\input{style/new_style}
\input{style/macros_SII}
\usepackage{multicol}
\usepackage{siunitx}
%\usepackage{picins}
\fichetrue
%\fichefalse

\proftrue
\proffalse

\tdtrue
%\tdfalse

\courstrue
\coursfalse


\def\classe{\textsf{PSI$\star$ -- MP}}
\def\xxnumpartie{Cycle 05}
\def\xxpartie{Modéliser le comportement des systèmes mécaniques dans le but d'établir une loi de comportement ou de déterminer des actions mécaniques en utilisant les méthodes énergétiques}

\def\xxnumchapitre{Chapitre 1 \vspace{.2cm}}
\def\xxchapitre{\hspace{.12cm} Approche énergétique}

\def\discipline{Sciences \\Industrielles de \\ l'Ingénieur}
\def\xxtete{Sciences Industrielles de l'Ingénieur}




\def\xxtitreexo{Application 01}%Motorisation du moteur Haibike}
\def\xxsourceexo{\hspace{.2cm} \footnotesize{Pôle Chateaubriand -- Joliot-Curie}}


\def\xxposongletx{2}
\def\xxposonglettext{1.45}
\def\xxposonglety{20}
%\def\xxonglet{Part. 1 -- Ch. 3}
\def\xxonglet{\textsf{Cycle 05}}

\def\xxactivite{Application}
\def\xxauteur{\textsl{Pôle Chateaubriand -- Joliot-Curie}}

\def\xxcompetences{%
\textsl{%
\textbf{Savoirs et compétences :}\\
%Les sources sont associées par un \emph{hacheur série}. La détermination des grandeurs électriques associées à ce montage permet de conclure vis à vis du cahier des charges.
%\noindent \textbf{Résoudre :} à partir des modèles retenus :
%\begin{itemize}[label=\ding{112},font=\color{ocre}] 
%\item choisir une méthode de résolution analytique, graphique, numérique;
%\item mettre en \oe{}uvre une méthode de résolution.
%\end{itemize}
%\begin{itemize}[label=\ding{112},font=\color{ocre}] 
%\item \textit{Rés -- C1.1 :} Loi entrée sortie géométrique et cinématique -- Fermeture géométrique.
%\end{itemize}
%
%\noindent \textit{Mod2 -- C4.1 :} Représentation par schéma bloc.
}}

\def\xxfigures{
\includegraphics[width=.6\linewidth]{images/fig_00}
}%figues de la page de garde


\def\xxpied{%
Cycle 05 -- Modélisation mécanique -- Énergétique\\% afin de valider leurs performances.\\
Chapitre 1 -- \xxactivite%
}

\setcounter{secnumdepth}{5}
%---------------------------------------------------------------------------

\usepackage{pgfplots}
\begin{document}
\def\pathfig{images}
%\chapterimage{png/Fond_Cin}
\input{style/new_pagegarde}
\vspace{4.5cm}
\pagestyle{fancy}
\thispagestyle{plain}

\def\columnseprulecolor{\color{ocre}}
\setlength{\columnseprule}{0.4pt} 

\def\pathfig{images}

\ifprof
%\begin{multicols}{2}
\else
\begin{multicols}{2}
\fi

On s’intéresse un dispositif d’ouverture des deux parties du vantail de gauche d’une porte automatique
schématisé ci-dessous :

\begin{center}
\includegraphics[width=\linewidth]{images/fig_01.png}
\end{center}

Le mécanisme étudié comprend essentiellement :
\begin{itemize}
\item un bâti 1 auquel est lié le repère $\repere{A}{x_1}{y_1}{z_1}$ supposé galiléen;
\item deux vantaux 2 et 3 de dimensions identiques et de masse identique $M$;
\item un châssis solidaire du vantail 2 et guidé en translation par rapport au bâti 1 ;
\item deux poulies 4 et 5, de même rayon $R$, en rotation par rapport au châssis lié à 2 ;
\item une courroie crantée dont les brins rectilignes, notés 6a et 6b, sont liés respectivement aux bâti 1 et
au vantail 5;
\item un moteur dont le stator est fixe sur le châssis lié à 2 et le rotor lié à la poulie 4.
\end{itemize}

On note :
\begin{itemize}
\item $g$, l‘accélération de la pesanteur orientée suivant $-\vect{z_1}$;
\item $C_ m$, le couple moteur exercé par le stator sur le rotor ;
\item $J$ , l’inertie équivalente, ramenée sur l’axe du moteur, de toutes les masses et inerties autres que
celles des deux vantaux.
\end{itemize}

On néglige la masse de la courroie.
On suppose toutes les liaisons parfaites et l’absence de glissement entre la courroie et les poulies.
\begin{obj}
Déterminer la loi entrée-sortie en effort  $C_m=f(M)$ en vue d’adapter les caractéristiques du moteur
électrique aux performances souhaitées.
\end{obj}

\subparagraph{}
\textit{Déterminer la relation entre $\omega_{42}$ et $\omega_{52}$ qui sont les vitesses de rotation respectives de 4/2 et 5/2.}
\ifprof
\begin{corrige}
\end{corrige}\else\fi

\subparagraph{}
\textit{Donner :
\begin{itemize}
\item la relation entre $\omega_{42}$ et $u_{21}$ la vitesse de déplacement du vantail 2 par rapport au bâti 1 ;
\item la relation entre $u_{21}$ et $u_{31}$ la vitesse de déplacement du vantail 3 par rapport au bâti 1.
\end{itemize}}
\ifprof
\begin{corrige}
\end{corrige}\else\fi

\subparagraph{}
\textit{Déterminer l’expression de l’énergie cinétique de l’ensemble $\Sigma=\left\{2,3,4,5\right\}$ dans son mouvement par rapport au bâti 1.}
\ifprof
\begin{corrige}
\end{corrige}\else\fi

\subparagraph{}
\textit{Appliquer le théorème de l’énergie cinétique à $\Sigma$ dans son mouvement par rapport au bâti 1. En
déduire la loi entrée-sortie en effort recherchée.}
\ifprof
\begin{corrige}
\end{corrige}\else\fi

\subparagraph{}
\textit{Interpréter les différents termes de l’expression trouvée.}
\ifprof
\begin{corrige}
\end{corrige}\else\fi

\vspace{1cm}
\footnotesize


\paragraph*{Éléments de correction}
%\begin{enumerate}
%\item $\omega_m=\dfrac{R_{p2}}{R_{p1}}\dfrac{V_{30}}{R_e}$ et $\omega_{\text{m max}}=\SI{351}{rad.s^{-1}}$.
%\item $C_m=\dfrac{1}{\eta}\left( T_{\text{coureur}\rightarrow 3} R_e \dfrac{R_{p1}}{R_{p2}}\right)$.
%\item $P\left(0\to1/0 \right)=\dfrac{1}{0,9}\left(T R_e \dfrac{R_{p1}}{R_{p2}} \right) \omega_{\text{max}}=\SI{1349}{W}$.
%\item ...
%\end{enumerate}


\begin{enumerate}
\item $\omega_{42}=\omega_{52}$.
\item $u_{21}=R\omega_{42}$ et $u_{31}=2u_{21}$.
\item $E_{c\; \Sigma/1} = \dfrac{1}{2}\left(5MR^2+J\right)\omega_{42}^2$.
\item $C_m=\left(5MR^2+J\right)\dot{\omega}_{42}$.
\end{enumerate}

\ifprof
%\end{multicols}
\else
\end{multicols}
\fi

\newpage

\begin{center}
\includegraphics[width=\linewidth]{images/cor_01.png}
\end{center}
\begin{center}
\includegraphics[width=\linewidth]{images/cor_02.png}
\end{center}
\begin{center}
\includegraphics[width=\linewidth]{images/cor_03.png}
\end{center}


\end{document}



\begin{center}
\includegraphics[width=\linewidth]{images/fig_01.png}
\end{center}


\subparagraph{}
\textit{}
\ifprof
\begin{corrige}
\end{corrige}\else\fi


