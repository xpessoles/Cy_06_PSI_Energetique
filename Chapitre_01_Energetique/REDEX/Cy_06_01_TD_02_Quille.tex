%%%% Paramétrage du TD %%%%
\def\xxactivite{ \ifprof \normalsize{TD-- Corrigé } \else  \ifcolle Colle \else TD 2\fi \fi} % \normalsize \vspace{-.4cm}
\def\xxauteur{\textsl{Xavier Pessoles}}


\def\xxnumchapitre{Chapitre ** \vspace{.2cm}}
\def\xxchapitre{\hspace{.12cm} Analyse des systèmes mécanique}

\def\xxcompetences{%
\vspace{-.5cm}
\footnotesize{
\textsl{%
\textbf{Savoirs et compétences :}\\
\vspace{-.2cm}
%\begin{itemize}[label=\ding{112},font=\color{ocre}] 
%\item Mod2.C18.SF1 : Déterminer l’énergie cinétique d’un solide, ou d’un ensemble de solides, dans son mouvement par rapport à un autre solide.
%\item Res1.C1.SF1 : Proposer une démarche permettant la détermination de la loi de mouvement.
%\item Mod1.C5.SF2 : Déterminer la puissance des actions mécaniques extérieures à un solide ou à un ensemble de solides, dans son mouvement rapport à un autre solide.
%\item Mod1.C5.SF3 : Déterminer la puissance des actions mécaniques intérieures à un ensemble de solides.
%\end{itemize}
}}}


\def\xxtitreexo{Poulie Redex\ifnormal $\star$ \else \fi \iftdifficile $\star\star\star$ \else \fi }
\def\xxsourceexo{\hspace{.2cm}}% \footnotesize{Concours Commun Mines Ponts 2014}}

\def\xxfigures{
%\includegraphics[width=.6\textwidth]{fig_01}
}%figues de la page de garde


\iflivret
\input{../../style/new_pagegarde}
\else
\input{../../style/new_pagegarde}
\fi
\setlength{\columnseprule}{.1pt}

\pagestyle{fancy}
\thispagestyle{plain}

\ifprof
\vspace{4cm}
\else
\vspace{4cm}
\fi

\def\columnseprulecolor{\color{ocre}}
\setlength{\columnseprule}{0.4pt} 

%%%%%%%%%%%%%%%%%%%%%%%

\setcounter{exo}{0}


\ifprof
%\begin{multicols}{2}
\else
\begin{multicols}{2}
\fi

\section*{Mise en situation}
\ifprof
\else
La poulie Redex est un train épicycloïdal compact, utilisable en réducteur, multiplicateur ou différentiel. Le module étudié se compose de trois éléments coaxiaux pouvant tourner à des vitesses différentes. 



\subparagraph{}\textit{Réaliser le schéma cinématique (non minimal) du système.}

\subparagraph{}\textit{Déterminer le rapport de transmission du système.}

\subparagraph{}\textit{Décrire les solutions d'étanchéité utilisées.}

\subparagraph{}\textit{Justifier le choix des liaisons entre 9 et 1.}

\subparagraph{}\textit{Justifier le choix des liaisons entre 3 et 1.}

\subparagraph{}\textit{Réaliser le graphe de structure.}

\subparagraph{}\textit{Déterminer le degré d'hyperstatisme. Proposer un modèle isostatique en gardant le même nombre liaisons.}

\subparagraph{}\textit{Proposer un matériau et un processus de fabrication pour la l'arbre 3.}

\end{multicols}

\begin{center}
\includegraphics[width=.7\textwidth]{images/redex}
\end{center}
